% !TeX root = RL_main.tex

\documentclass[a4paper,twocolumn]{article}
%\documentclass{article}

%% Language and font encodings
\usepackage[english]{babel}
%\usepackage{times} % make the section title in Times New Roman

\usepackage{booktabs}
\usepackage{tabu}
\usepackage[T1]{fontenc}

\usepackage[a4paper,top=2.5cm,bottom=2cm,left=1.7cm,right=1.7cm,marginparwidth=1.75cm]{geometry}

%\usepackage[utf8]{inputenc}
\usepackage{amsmath}
\usepackage{lipsum} % to generate some filler text
%\usepackage{fullpage}
\usepackage{mathrsfs}
\usepackage{svg}
\usepackage{xr}
\usepackage{amsthm,cite,url,graphicx,booktabs,lipsum,color,bm,subcaption,soul}
%\usepackage[labelformat=empty,labelsep=none]{caption}
\usepackage{caption}
\usepackage{pifont,tikz,paralist,multirow,amssymb}
\usepackage{xifthen}
\usepackage{enumerate}
\usepackage{titlesec}
\newcounter{reviewer}
\setcounter{reviewer}{0}
\newcounter{point}[reviewer]
\setcounter{point}{0}

\usepackage{caption}
\captionsetup[table]{labelsep=period, labelfont=bf, justification=centering, singlelinecheck=off}

%\renewcommand{\figurename}{Fig.}
\captionsetup[figure]{labelfont={rm},name={Fig.},labelsep=period}

\usepackage{algorithm}
\usepackage{algorithmic}
\renewcommand{\algorithmicrequire}{\textbf{Input:}}
\renewcommand{\algorithmicensure}{\textbf{Output:}}



% This refines the format of how the reviewer/point reference will appear.
%\renewcommand{\thepoint}{C\,\thereviewer.\arabic{point}}

\renewcommand{\thepoint}{Comment\,\thereviewer.\arabic{point}}

% command declarations for reviewer points and our responses

\newenvironment{point1}
{\refstepcounter{point} \bigskip \noindent {\textbf{Associate Editor Comment~}} ---\ }
{\par }

\newenvironment{pointGen}
{\refstepcounter{point} \bigskip \noindent {\textbf{Reviewer General Comment~}} ---\ }
{\par }

\newcommand{\reviewersection}{\stepcounter{reviewer} \bigskip \hrule
	\section*{Reviewer \thereviewer}}

% \newenvironment{point}

%    {\refstepcounter{point} \bigskip \noindent {\textbf{Reviewer~\thepoint} } ---\ }
%    {\par }

\newenvironment{point}
{\refstepcounter{point} \bigskip \noindent {\textbf{Reviewer~\thepoint} } ---\ }
{\par }


\newcommand{\shortpoint}[1]{\refstepcounter{point}   \bigskip \noindent 
	{\textbf{Reviewer~Point~\thepoint} } ---~#1  }

\newenvironment{reply}
{\medskip \noindent \color{blue} \begin{sf}\textbf{Reply}:\  }
	{\medskip \par \end{sf}}


\newenvironment{append}
{\bigskip \noindent {\textbf{Appendix~} } ---\ }
{\par }

\newcommand{\shortreply}[2][]{\medskip \noindent \begin{sf}\textbf{Reply}:\  #2
		\ifthenelse{\equal{#1}{}}{}{ \hfill \footnotesize (#1)}%
		\medskip \end{sf}}

\makeatletter
\newcommand*{\addFileDependency}[1]{% argument=file name and extension
	\typeout{(#1)}
	\@addtofilelist{#1}
	\IfFileExists{#1}{}{\typeout{No file #1.}}
}
\makeatother

\newcommand*{\myexternaldocument}[1]{%
	\externaldocument{#1}%
	\addFileDependency{#1.tex}%
	\addFileDependency{#1.aux}%
}
\def\TagSSST{9}
\def\TagSSSTTV{10}

\def\tagFigCaseOne{2}
\def\tagFigCaseThree{3}
\def\tagFigCaseFour{4}
\def\tagFigCaseFive{5}
\def\tagFigCaseEight{6}
\def\tagFigSAMMap{7}
\def\tagFigConvAnal{10}
\def\tagFigParamAnal{11}
\def\tagFigDiff{12}

\def\tagTabComp{2}
\def\tagTabMPSNR{3}
\def\tagTabMSSIM{4}
\def\tagHyperparam{5}





\def\thesubsubsection{\arabic{subsubsection}}
\titleformat{\subsubsection}[runin]
{\normalfont\bfseries}
{\indent\thesubsubsection)}{0.5em}{}
\title{Reply Letter for GRSL-00116-2025}

\author{%
	Yudai Inada, Shingo Takemoto, Shunsuke Ono\\
	Institute of Science Tokyo\\
	inada.y.de71@m.isct.ac.jp, takemoto.s.e908@m.isct.ac.jp, ono.s.5af2@m.isct.ac.jp\\
}


\begin{document}
\onecolumn
\maketitle
\begin{flushleft}
	Paper Number: GRSL-00116-2025\\
	Title: Efficient and Accurate Full-Waveform Inversion with Total Variation Constraint\\
	Date of Evaluation: 14-Jan-2025.\\
	Date for Revision: 28-May-2024.\\
\end{flushleft}

We would like to take this opportunity to thank Dr. Xiuping Jia
for kindly handling our manuscript and the reviewers for their carefully reading of our manuscript and giving useful comments.
Furthermore, we also appreciate the recommendation to resubmit to the Journal of Selected Topics in Applied Earth Observations and Remote Sensing.
We have carefully studied all the comments given by the associate editor and the reviewers and have made revisions,
which we hope will meet with your approval.
Please kindly see the following pages that contain our detailed responses to each comment raised by the associate editor and each reviewer.
We have also uploaded a PDF file of the revised manuscript,
where we changed the text color to blue in revised parts.

All numberings (of sections, equations, references, etc.) in the original comments given by the reviewers refer to the previous manuscript, and those in the following replies refer to the revised manuscript. We have highlighted significant changes in the revised manuscript and the replies in \textcolor{blue}{blue}.
% The numberings of sections and equations in the following answers refer to the revised manuscript,
% and all the references in the following answers are given at the end of the letter.

We are grateful for your help and comments, which enabled us to significantly improve the quality of our paper.
We hope that our response is sufficient to make our manuscript suitable for publication in the IEEE Journal of Selected Topics in Applied Earth Observations and Remote Sensing.

\twocolumn
\clearpage
%	\section*{Notes on revision made to manuscript Paper $\#$ABC-2099-12-1234
%	}
%	\vspace{10pt}
%	
%	The authors would like to thank the editor and reviewers for their constructive comments and suggestions that have helped improve the quality of this manuscript. The manuscript has undergone a thorough revision according to the editor and reviewers’ comments. Please see below our responses. For the reviewers’ convenience, we have highlighted significant changes in the revised manuscript in \textcolor{blue}{blue}.
	
% Let's start point-by-point with Reviewer 1

\onecolumn
\section*{Response to the Associate Editor}
% General intro text goes here
\begin{point1}
(There are no comments.)
\end{point1}


\begin{reply}
Thank you very much for handling our paper carefully. Following the instruction, we have provided an item-by-item response to the reviewer's comments. Although the details are described in the answers to each reviewer's comment, we would like to summarize the main revisions in brief:

\begin{itemize} 
	\item We have reorganized the paragraphs in the Introduction for improved readability. (For the reply to Comment 1.1)
	\item We have improved the description of the proposed method. (For the replies to Comments 1.5, 2.1, 2.3, and 3.1)
	\item We have added references and comparative methods. (For the replies to Comments 1.2, 1.3, 1.4, 2.6, 3.2, and 3.5)
	\item We have conducted additional experiments with additional noise conditions. (For the replies to Comments 1.4, 2.5, and 3.4)
	\item We have expanded the experiments by incorporating an additional dataset. (For the reply to Comment 1.5)
	\item We have conducted further analysis of the proposed method. (For the replies to Comments 2.3 and 2.7)
\end{itemize}

%\begin{itemize}
%	\item We have added additional references. (For the replies to Comment 2.1, 2.2, and 3.1)
%	\item We have rewritten the contributions to be clearer. (For the replies to Comment 1.1, 1.2, 1.3, 1.4, 3.2, and 3.3)
%	\item We have improved the description of our method. (For the replies to Comment 1.1, 3.2, and 3.3)
%	\item We have performed substantial additional experiments. (For the replies to Comment 2.4, 2.5, 3.4, 3.5, 3.6, 3.7)
%\end{itemize}
\end{reply}


\twocolumn
\clearpage

\twocolumn[
\section*{Response to Reviewers}
]

%%%%%%%%%%%%%%%%%%%%%%%%%%%%%%%%%%%%%%%%%%%%%%%%%%%
\reviewersection
% General intro text goes here

\begin{pointGen}
	This paper presents an implementation of TV regularization for solving the FWI problem using the classical reduced-space formulation with an L2-norm misfit function. The author introduces the primal-dual splitting method with proximal operators to eliminate the inner loop required for nonlinear TV mapping. While the proposed algorithm may be useful for certain inverse problems, I find it unsuitable for FWI due to the following reasons:
\end{pointGen}

\begin{reply}
	Thank you for your careful reading and fruitful comments. (UNC)
\end{reply}

\setcounter{point}{0}

%%%%%%%%%%%%%%%%
\begin{point}
	The first sentence states: "This paper proposes a computationally efficient algorithm to address the Full-Waveform Inversion (FWI) problem with a Total Variation (TV) constraint …"
	This claim is misleading. While the proposed method may be beneficial for linear inverse problems, where gradient calculations are inexpensive and the computational cost is dominated by TV regularization, this is not the case for FWI. In FWI, the leading-order cost is dominated by gradient computation, which primarily stems from solving the PDEs. Consequently, the overhead of implementing TV regularization, even with an inner loop, is negligible. In other words, the inner loop for TV regularization is not a computational bottleneck in FWI.
\end{point}

% Our reply
\begin{reply}
AAA

\end{reply}

\begin{point}
	The proposed method is based on the reduced-space approach, which is known to be highly sensitive to cycle-skipping. Over the past decade, extensive research has demonstrated the advantages of extended-source FWI using Alternating Direction Method of Multipliers (ADMM), which enables TV regularization to be applied efficiently without inner loops. The paper does not adequately compare or contrast the proposed approach with these state-of-the-art methods.
\end{point}

\begin{reply}
AAA

\end{reply}
\begin{point}
	The numerical example provided is not representative of realistic FWI case. The problem is formulated on a small subset of a benchmark model, placing the inverse problem in a linear regime. Furthermore, as observed in Figure 5, the method requires 5000 iterations, which is highly impractical and significantly deviates from the efficiency of modern FWI algorithms.
\end{point}

\begin{reply}
	AAA
	
\end{reply}
\begin{point}
	Given these limitations, I believe the proposed method for implementing TV regularization would be better suited for a different class of inverse problems rather than FWI. I recommend the author reconsider the application domain and provide a more thorough comparison with existing state-of-the-art methods.
\end{point}

\begin{reply}
	AAA
	
\end{reply}

% 1-4?

%\section*{Appendix: Experimental Settings}
Appendix.
\begin{thebibliography}{100}
	\bibitem[16]{Zhuang2023FastHyMix} L. Zhuang and M. K. Ng, ``FastHyMix: Fast and paramter-free hyperspectral image mixed noise removal,'' \textit{IEEE Trans. Neural Netw. Learn. Syst.,} vol. 34, no. 8, pp. 4702-4716, 2023.
	
	\bibitem[17]{Peng2024RCILD} J. Peng, H. Wang, X. Cao, Q. Zhao, J. Yao, H. Zhang, and D. Meng, ``Learnable Representative Coefficient Image Denoiser for Hyperspectral Image,'' \textit{IEEE Trans. Geosci. Remote Sens.}, vol. 62, pp. 1-16, 2024.
	
	\bibitem[25]{Aggarwal2016SSTV} H. K. Aggarwal and A. Majumdar, ``Hyperspectral Image Denoising Using Spatio-Spectral Total Variation,'' \textit{IEEE Geosci. Remote Sens. Lett.}, vol.13, no. 3, pp. 442--446, 2016.
	
	\bibitem[28]{Takeyama2020HSSTV} S. Takeyama, S. Ono, and I. Kumazawa, ``A Constrained Convex Optimization Approach to Hyperspectral Image Restoration with Hybrid Spatio-Spectral Regularization,'' \textit{Remote Sens.}, vol. 12, no. 21, 2020.
	
	\bibitem[29]{Wang2021l0l1HTV} M. Wang, Q. Wang, J. Chanussot, and D. Hong, ``$l_0$-$l_1$ Hybrid Total Variation Regularization and its Applications on Hyperspectral Image Mixed Noise Removal and Compressed Sensing,'' \textit{IEEE Trans. Geosci. Remote Sens.}, vol. 59, no. 9, pp. 7695-7710, 2021. 
	
	\bibitem[34]{Wang2018LRTDTV} Y. Wang, J. Peng, Q. Zhao, Y. Leung, X. Zhao, and D. Meng, ``Hyperspectral Image Restoration Via Total Variation Regularized Low-Rank Tensor Decomposition,'' \textit{IEEE J. Sel. Topics Appl. Earth Observ. Remote Sens.}, vol. 11, no. 4, pp. 1227-1243, 2018.
	
	\bibitem[35]{Chen2020LRTDGS} Y. Chen , W. He, N. Yokoya, and T.-Z. Huang, ``Hyperspectral Image Restoration Using Weighted Group Sparsity-Regularized Low-Rank Tensor Decomposition,'' \textit{IEEE Trans. Cybern.}, vol. 50, no. 8, pp. 3556-3570, 2020.
	
	\bibitem[36]{Chen2022FGSLR} Y. Chen, T. Huang, W. He, X. Zhao, H. Zhang, and J. Zeng, ``Hyperspectral Image Denoising Using Factor Group Sparsity-Regularized Nonconvex Low-Rank Approximation,'' \textit{IEEE Trans. Geosci. Remote Sens.}, vol. 60, pp. 1-16, 2022.
	
	\bibitem[37]{Chen2023TPTV} Y. Chen, W. Cao, L. Pang, J. Peng, and X. Cao, ``Hyperspectral Image Denoising Via Texture-Preserved Total Variation Regularizer,'' \textit{IEEE Trans. Geosci. Remote Sens.}, vol. 61, pp. 1-14, 2023.	
	
	\bibitem[38]{Lefkimmiatis2015STV} S. Lefkimmiatis, A. Roussos, P. Maragos, and M. Unser, ``Structure Tensor Total Variation,'' \textit{SIAM J. Imag. Sci.}, vol. 8, no. 2, pp. 1090--1122, 2015. 
	
	\bibitem[41]{Kurihara2017SSST} R. Kurihara, S. Ono, K. Shirai, and M. Okuda, ``Hyperspectral image restoration based on spatio-spectral structure tensor regularization,'' \textit{Proc. Eur. Signal Process. Conf. (EUSIPCO)}, pp. 488-492, 2017.
	
	\bibitem[42]{Forstner1987Fast} W. F{\"o}rstner and E. G{\"u}lch, ``A fast operator for detection and precise location of distinct points, corners and centres of circular features,'' \textit{Proc. ISPRS Intercommission Conf. Fast Process. Photogramm. Data}, vol. 6, pp. 281--305, 1987.
	
	\bibitem[43]{Weickert1998Anisotropic} J. Weickert, ``Anisotropic diffusion in image processing,'' \textit{Stuttgart, Germany: B. G. Teubner}, vol. 1, 1998.
	
	\bibitem[44]{Jahne2005Digital} B. J{\"a}hne, ``Digital image processing,'' \textit{Springer Sci. Bus. Media}, 2005.
	
	\bibitem[52]{Pock2011PPDS} T. Pock and A. Chambolle, ``Diagonal preconditioning for first order primal-dual algorithms in convex optimization,'' \textit{Proc. IEEE Int. Conf. Comput. Vis. (ICCV)}, pp. 1762--1769, 2011.
	
	\bibitem[60]{Wang2004SSIM} Z. Wang, A. Bovik, H. Sheikh, and E. Simoncelli, ``Image quality assessment: from error visibility to structural similarity,'' \textit{IEEE Trans. Image Process.}, vol. 13, no. 4, pp. 600-612, 2004.
	
	
	
%	\bibitem[61]{Matsaglia1974Equalities} G. Matsaglia and G. PH Styan, ``Equalities and inequalities for ranks of matrices,'' \textit{Linear multilinear Algebra}, vol. 2, no. 3, pp. 269--292, 1974.
	
	
\end{thebibliography}
%%%%%%%%%%%%%%%%





\clearpage
%%%%%%%%%%%%%%%%%%%%%%%%%%%%%%%%%%%%%%%%%%%%%%%%%%%
\reviewersection


\begin{pointGen}
	I appreciate the opportunity to review the manuscript titled "Efficient and Accurate Full-Waveform Inversion with Total Variation Constraint." While I find the core idea interesting and relevant to the field, the manuscript is currently far from being acceptable for publication and requires significant revisions. Below are my key concerns:
\end{pointGen}

\begin{reply}
BBB
\end{reply}

\setcounter{point}{0}

%%%%%%%%%%%%%%%%
\begin{point}
	The introduction lacks a clear differentiation between the proposed method and existing TV-regularized FWI approaches, particularly those based on splitting techniques. The authors should explicitly state how their method improves upon prior work (or its connections), such as ADMM-based, split Bregman, FISTA, and proximal Newton approaches.
\end{point}

\begin{reply}
AAA
\end{reply}


\begin{point}
	Some statements in the introduction are misleading. For instance, TV regularization is well-suited for piecewise constant models, not piecewise smooth ones, as claimed in the manuscript. The authors should correct such inaccuracies.
\end{point}

\begin{reply}
	AAA
\end{reply}


\begin{point}
	The theoretical formulation requires further clarification. Certain concepts are presented without adequate justification.
\end{point}

\begin{reply}
	AAA
\end{reply}

% どこのこと言ってるかわからん
% ほかの返答に合わせてこじつけられそう
\begin{point}
	The tuning of the regularization parameter in the proposed method remains unclear. The authors should discuss how their approach overcomes known challenges related to parameter selection in TV-regularized FWI.
\end{point}

\begin{reply}
	AAA
\end{reply}


\begin{point}
	The design of numerical tests does not convincingly demonstrate the advantages of the proposed method. The authors should test their method on more challenging models.
\end{point}

\begin{reply}
	AAA
\end{reply}


\begin{point}
	The model size used in the experiments is quite small. The authors should justify their choice and, test their method on larger models. Also, the data should contain large offset data that are critical for FWI.
\end{point}

\begin{reply}
	AAA
\end{reply}


\begin{point}
	A key missing element is the residual curve  for both standard FWI and TV-FWI. This would provide insights into the convergence behavior of the proposed method.
\end{point}

\begin{reply}
	AAA
\end{reply}


\begin{point}
	Given that FWI typically relies on L-BFGS for velocity updates rather than simple gradient descent, the authors should discuss whether their PDS method can be adapted for L-BFGS.
\end{point}

\begin{reply}
	AAA
\end{reply}



%\section*{Appendix: Experimental Settings}
As ground-truth HS images, we adopt three HS image dataset: \textit{Jasper Ridge}\footnote{\url{https://rslab.ut.ac.ir/data}} cropped to size $100 \times 100 \times 198$, \textit{Pavia University}\footnote{\url{https://www.ehu/ccwintco/index/php/Hyperspectral_Remote_Sensing_Scenes}} cropped to size $120 \times 120 \times 98$, and \textit{Beltsville}\footnote{\url{https://www.spectir.com/contact#free-data-samples}} cropped to size $100 \times 100 \times 128$.
All the intensities of three HS images were normalized within the range $[0, 1]$.

HS images are often degraded by a mixture of various types of noise in real-world scenarios.
Thus, in the experiments, we consider the following eight cases of noise contamination:
\begin{itemize}
	\setlength{\leftskip}{18pt}
	\item [\revise{Case 1:}] \revise{The observed HS image is contaminated by only white Gaussian noise with the standard deviation $\StanDevGauss = 0.05$.}
	\item [Case 2:] The observed HS image is contaminated by white Gaussian noise with the standard deviation $\StanDevGauss = 0.05$ and salt-and-pepper noise with the rate $\RateSparse = 0.05$.
	\item [Case 3:] The observed HS image is contaminated by white Gaussian noise with the standard deviation $\StanDevGauss = 0.1$ and salt-and-pepper noise with the rate $\RateSparse = 0.05$.
	\item [\revise{Case 4:}] \revise{The observed HS image is contaminated by only vertical stripe noise whose intensity is uniformly random in the range $[-0.5, 0.5]$ with the rate $\RateStripe = 0.05$.}
	\item [Case 5:] The observed HS image is contaminated by white Gaussian noise with the standard deviation $\StanDevGauss = 0.05$ and vertical stripe noise whose intensity is uniformly random in the range $[-0.5, 0.5]$ with the rate $\RateStripe = 0.05$.
	\item [Case 6:] The observed HS image is contaminated by white Gaussian noise with the standard deviation $\StanDevGauss = 0.1$ and vertical stripe noise whose intensity is uniformly random in the range $[-0.5, 0.5]$ with the rate $\RateStripe = 0.05$.
	\item [Case 7:] The observed HS image is contaminated by white Gaussian noise with the standard deviation $\StanDevGauss = 0.05$, salt-and-pepper noise with the rate $\RateSparse = 0.05$, and vertical stripe noise whose intensity is uniformly random in the range $[-0.5, 0.5]$ with the rate $\RateStripe = 0.05$.
	\item [Case 8:] The observed HS image is contaminated by white Gaussian noise with the standard deviation $\StanDevGauss = 0.1$, salt-and-pepper noise with the rates $\RateSparse = 0.05$, and vertical stripe noise whose intensity is uniformly random in the range $[-0.5, 0.5]$ with the rates $\RateStripe = 0.05$.
\end{itemize}

The block size of $\SSSTTV$ was set to $10 \times 10 \times \NumBand$.
The radii $\RadiusSparse$, $\RadiusStripe$, and $\RadiusFidel$ were set as follows:
\begin{equation}
	\label{eq:RL_1_Expt_RadiusSet}
	\RadiusSparse = \ParamsRadius \tfrac{\NumAll \RateSparse}{2}, \:
	\RadiusStripe = \ParamsRadius \tfrac{0.5 \NumAll \RateStripe (1 - \RateSparse)}{2}, \: \RadiusFidel = \ParamsRadius \sqrt{\StanDevGauss^2 \NumAll (1 - \RateSparse)},
\end{equation}
where the parameter $\ParamsRadius$ was set to $0.95$. The hyperparameters, including the comparison methods, are shown in Table~\ref{tab:RL1_6_HyperParam}.
For specific cases, adjustments were made to improve the accuracy of the parameter settings. In Case 1, where only Gaussian noise is present, the noise concentrates more on the corresponding term compared to mixed noise cases. To reflect this, $\ParamsRadius$ was set to $0.98$. Similarly, in Case 4, where only Stripe noise is present, $\ParamsRadius$ was also set to $0.98$ for the same reason. Furthermore, in Case 4, the fidelity term $\| v - u - t \|_{2}$ becomes zero, which can lead to instability in the solution. To address this, $\RadiusFidel$ was fixed to $0.01$ to ensure stability.
The stopping criterion of Alg.~1 were set as follows:
\begin{equation}
	\label{eq:RL_1_Expt_StopCri_simulated}
	\frac{\| \HSIClean^{(\IndexAlg+1)} - \HSIClean^{(\IndexAlg)} \|_2}{\| \HSIClean^{(\IndexAlg)}\|_{2}} < 1.0 \times 10^{-5}.
\end{equation}


For the quantitative evaluation, we employed the mean peak signal-to-noise ratio (MPSNR):
\begin{equation}
	\label{eq:RL_1_Expt_MPSNR}
	\mathrm{MPSNR} = \frac{1}{\NumBand} \sum_{\IndexBand=1}^{\NumBand} 10\log_{10}\frac{\NumVert \NumHori}{\|\HSIClean_{\IndexBand} - \bar{\HSIClean}_{\IndexBand}\|_{2}^{2}},
\end{equation}
and the mean structural similarity index (MSSIM)~\cite{Wang2004SSIM}:
\begin{equation}
	\label{eq:RL_1_Expt_MSSIM}
	\mathrm{MSSIM} = \frac{1}{\NumBand} \sum_{\IndexBand=1}^{\NumBand} \mathrm{SSIM}(\HSIClean_{\IndexBand}, \bar{\HSIClean}_{\IndexBand}),
\end{equation}
where $\HSIClean_{\IndexBand}$ and $\bar{\HSIClean}_{\IndexBand}$ are the $\IndexBand$-th band of the ground true HS image $\HSIClean$ and the estimated HS image $\bar{\HSIClean}$, respectively.
Generally, higher MPSNR and MSSIM values are corresponding to better denoising performances. Because the boundary conditions are circulant, we evaluate them by cutting off the first and last three bands. 

\begin{thebibliography}{100}
	\bibitem[16]{Zhuang2023FastHyMix} L. Zhuang and M. K. Ng, ``FastHyMix: Fast and paramter-free hyperspectral image mixed noise removal,'' \textit{IEEE Trans. Neural Netw. Learn. Syst.,} vol. 34, no. 8, pp. 4702-4716, 2023.
	
%	\bibitem[17]{Peng2024RCILD} J. Peng, H. Wang, X. Cao, Q. Zhao, J. Yao, H. Zhang, and D. Meng, ``Learnable Representative Coefficient Image Denoiser for Hyperspectral Image,'' \textit{IEEE Trans. Geosci. Remote Sens.}, vol. 62, pp. 1-16, 2024.
	
	\bibitem[25]{Aggarwal2016SSTV} H. K. Aggarwal and A. Majumdar, ``Hyperspectral Image Denoising Using Spatio-Spectral Total Variation,'' \textit{IEEE Geosci. Remote Sens. Lett.}, vol.13, no. 3, pp. 442-446, 2016.
	
	\bibitem[28]{Takeyama2020HSSTV} S. Takeyama, S. Ono, and I. Kumazawa, ``A Constrained Convex Optimization Approach to Hyperspectral Image Restoration with Hybrid Spatio-Spectral Regularization,'' \textit{Remote Sens.}, vol. 12, no. 21, 2020.
	
	\bibitem[29]{Wang2021l0l1HTV} M. Wang, Q. Wang, J. Chanussot, and D. Hong, ``$l_0$-$l_1$ Hybrid Total Variation Regularization and its Applications on Hyperspectral Image Mixed Noise Removal and Compressed Sensing,'' \textit{IEEE Trans. Geosci. Remote Sens.}, vol. 59, no. 9, pp. 7695-7710, 2021. 
	
	\bibitem[34]{Wang2018LRTDTV} Y. Wang, J. Peng, Q. Zhao, Y. Leung, X. Zhao, and D. Meng, ``Hyperspectral Image Restoration Via Total Variation Regularized Low-Rank Tensor Decomposition,'' \textit{IEEE J. Sel. Topics Appl. Earth Observ. Remote Sens.}, vol. 11, no. 4, pp. 1227-1243, 2018.
	
%	\bibitem[35]{Chen2020LRTDGS} Y. Chen , W. He, N. Yokoya, and T.-Z. Huang, ``Hyperspectral Image Restoration Using Weighted Group Sparsity-Regularized Low-Rank Tensor Decomposition,'' \textit{IEEE Trans. Cybern.}, vol. 50, no. 8, pp. 3556-3570 2020.
	
	\bibitem[36]{Chen2022FGSLR} Y. Chen, T. Huang, W. He, X. Zhao, H. Zhang, and J. Zeng, ``Hyperspectral Image Denoising Using Factor Group Sparsity-Regularized Nonconvex Low-Rank Approximation,'' \textit{IEEE Trans. Geosci. Remote Sens.}, vol. 60, pp. 1-16, 2022.
	
	\bibitem[37]{Chen2023TPTV} Y. Chen, W. Cao, L. Pang, J. Peng, and X. Cao, ``Hyperspectral Image Denoising Via Texture-Preserved Total Variation Regularizer,'' \textit{IEEE Trans. Geosci. Remote Sens.}, vol. 61, pp. 1-14, 2023.	
	
	\bibitem[38]{Lefkimmiatis2015STV} S. Lefkimmiatis, A. Roussos, P. Maragos, and M. Unser, ``Structure Tensor Total Variation,'' \textit{SIAM J. Imag. Sci.}, vol. 8, no. 2, pp. 1090--1122, 2015. 
	
	\bibitem[41]{Kurihara2017SSST} R. Kurihara, S. Ono, K. Shirai, and M. Okuda, ``Hyperspectral image restoration based on spatio-spectral structure tensor regularization,'' \textit{Proc. Eur. Signal Process. Conf. (EUSIPCO)}, pp. 488-492, 2017.
	
%	\bibitem[42]{Forstner1987Fast} W. F{\"o}rstner and E. G{\"u}lch, ``A fast operator for detection and precise location of distinct points, corners and centres of circular features,'' \textit{Proc. ISPRS Intercommission Conf. Fast Process. Photogramm. Data}, vol. 6, pp. 281--305, 1987.
	
%	\bibitem[43]{Weickert1998Anisotropic} J. Weickert, ``Anisotropic diffusion in image processing,'' \textit{Stuttgart, Germany: B. G. Teubner}, vol. 1, 1998.
	
%	\bibitem[44]{Jahne2005Digital} B. J{\"a}hne, ``Digital image processing,'' \textit{Springer Sci. Bus. Media}, 2005.
	
	\bibitem[52]{Pock2011PPDS} T. Pock and A. Chambolle, ``Diagonal preconditioning for first order primal-dual algorithms in convex optimization,'' \textit{Proc. IEEE Int. Conf. Comput. Vis. (ICCV)}, pp. 1762--1769, 2011.
	
	\bibitem[60]{Wang2004SSIM} Z. Wang, A. Bovik, H. Sheikh, and E. Simoncelli, ``Image quality assessment: from error visibility to structural similarity,'' \textit{IEEE Trans. Image Process.}, vol. 13, no. 4, pp. 600-612, 2004.
	
	
	\bibitem[61]{Matsaglia1974Equalities} G. Matsaglia and G. PH Styan, ``Equalities and inequalities for ranks of matrices,'' \textit{Linear multilinear Algebra}, vol. 2, no. 3, pp. 269--292, 1974.
	
	
\end{thebibliography}

%%%%%%%%%%%%%%%%



\clearpage
%%%%%%%%%%%%%%%%%%%%%%%%%%%%%%%%%%%%%%%%%%%%%%%%%%%
\reviewersection


\begin{pointGen}
	The manuscript presents a novel algorithm for full-waveform inversion (FWI) that incorporates a Total Variation (TV) constraint into the inversion process via a primal-dual splitting (PDS) method. The goal is to improve both the computational efficiency and the accuracy of subsurface reconstructions. Overall, the paper addresses an important challenge in seismic imaging, and the proposed method is of potential interest to the geophysical community. However, several aspects require further clarification and refinement before publication. I therefore recommend major revisions.
\end{pointGen}

\begin{reply}
	BBB
\end{reply}

\setcounter{point}{0}

%%%%%%%%%%%%%%%%
\begin{point}
	While the approach is innovative, the manuscript would benefit from a more explicit comparison with existing TV-constrained FWI methods (e.g., references [21–24]). It is not entirely clear how the proposed formulation and algorithm differ from these earlier approaches beyond the elimination of inner loops. A discussion emphasizing the advantages—both theoretical and practical—would help contextualize your contribution.
\end{point}

\begin{reply}
	AAA
\end{reply}

% 比較できるならする
% できなそうなら、紙面の都合上、って感じで逃げるか。
% もしくは記述を厚くしますという説もありけり

\begin{point}
	The selection of the TV bound parameter $\alpha$ (ranging from 100 to 700) and the step sizes ($\gamma_{1}$ and $\gamma_{2}$) is not sufficiently justified. Given that FWI is highly sensitive to such hyperparameters, it would be beneficial to include a discussion on how these parameters could be selected in practice, possibly through data-driven approaches or based on prior geological knowledge.
\end{point}

\begin{reply}
	AAA
\end{reply}

% 2-4優先


\begin{point}
	Although the algorithm appears efficient, more details on its convergence behavior are needed. For example, how many iterations does the method typically require to converge relative to standard FWI? Is there any guarantee (theoretical or empirical) on the stability of the inversion, particularly when noisy data are used?
\end{point}

\begin{reply}
	AAA
\end{reply}


\begin{point}
	The use of a box constraint to keep velocity values within [l, u] is common; however, further discussion of the physical implications of the TV constraint would enhance the paper. For instance, how does enforcing TV relate to known geological structures or stratigraphic features? Can you provide examples or discuss cases where the TV constraint may excessively smooth features that are important for interpretation?
\end{point}

\begin{reply}
	AAA
\end{reply}

% 3章のalphaのところで詳しく説明すべき?
% 実験の章のところでやってます、みたいな感じでいい気がする
% わかりやすくするために、\subsubsubsectionみたいなのを使ってもよいのか、?
\begin{point}
	Although the manuscript claims improved computational efficiency, it lacks quantitative runtime comparisons. Including a table or a discussion of the actual computational times relative to a conventional FWI implementation would significantly strengthen this claim.
\end{point}

\begin{reply}
	AAA
\end{reply}


\begin{point}
	The figures depicting velocity models (Figs. 3 and 4) are central to evaluating the performance. To improve these figures, please provide individual color bars for each figure and discuss in more detail the “wave-like artifacts” observed in standard FWI and how the TV constraint mitigates these artifacts. Additionally, please provide a more detailed discussion of the “point artifacts” near the surface, which appear similar to source-receiver artifacts; note that the number of these artifacts (four in Fig. 3a) is much smaller than the number of sources (20) or receivers (101). Finally, please change the labeling in Fig. 4 from (e–h) to (a–d), as Figs. 3 and 4 are two separate figures, not one combined figure.
\end{point}

\begin{reply}
	AAA
\end{reply}

% Color barを両方の図に載せよう
% artifactがなにかわかりづらいなら、矢印を使うのはあり


\begin{point}
	The conclusion could benefit from a more critical discussion of the limitations of the proposed approach. For example, under what conditions might the TV constraint be too restrictive? What future work is planned to extend or validate the method further?
\end{point}

\begin{reply}
	AAA
\end{reply}

% future workをconclusionに書くのが丸そうか
\begin{point}
	The definitions and descriptions of the variables should be improved. For example, the operator D appears in equation (2) without any definition; please clarify its meaning.
\end{point}

\begin{reply}
	AAA
\end{reply}

% Dの定義を書こう、 forward difference operatorぐらい
% 他もないか確認
\begin{point}
	Since the authors plan to share their source code, additional documentation—including replacing Japanese with English—is necessary. I attempted to understand the code but found it difficult due to insufficient guidance.
\end{point}

\begin{reply}
	AAA
\end{reply}



%\input{./RL_src/RL_3_Appendix}
%\input{./RL_src/RL_3_bib}

%%%%%%%%%%%%%%%%







\end{document}


