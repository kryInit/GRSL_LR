\IEEEPARstart{F}{ull-Waveform} Inversion (FWI)~\cite{FWI0,FWI1} is a technique for reconstructing subsurface properties from seismic data measured at multiple observation points.
The subsurface properties obtained through FWI are essential in geological research and resource exploration, such as locating gas and oil reservoirs, characterizing mineral deposits, and assessing groundwater flow patterns~\cite{FWI1,FWIApplicationGroundwater0,FWIApplicationGroundwater1}.
In addition to the geological field, FWI has also been successfully applied to non-destructive testing, including brain tissue analysis in medical imaging and material detection in industrial inspection ~\cite{FWIApplicationNonDestructiveTesting0,FWIApplicationNonDestructiveTesting1}.

FWI faces significant challenges in directly reconstructing subsurface properties from seismic data because of the nonlinear wave propagation and the inherent complexity of the observation process~\cite{FWI1}.
To address this issue, FWI is typically formulated as an optimization problem that minimizes the squared error between observed and simulated seismic data~\mbox{\cite{FWI0,CustomFWI0,CustomFWI1,CustomFWI2,CustomFWI3,CustomFWI4,CustomFWI5}}, which serves as the standard approach.
Nevertheless, the inherent ill-posedness of FWI necessitates the incorporation of regularization techniques.
Among these, Tikhonov regularization~\cite{tikhonov} and Total Variation (TV)-based methods~\cite{TV,TGV} are widely used to promote piecewise smoothness in the reconstructed subsurface properties, improving stability and accuracy~\cite{FWI-with-tikhonov-regularization,FWI-with-TV-regularization,FWI-with-directional-TV-regularization,FWI-with-high-order-TV-regularization,FWI-with-TGV-regularization}.
While these regularization techniques are effective, they often involve the careful tuning of a balance parameter that determines the trade-off between the FWI objective function and the regularization term.
To overcome this limitation, an alternative approach has been proposed: incorporating the TV prior as a constraint rather than as a regularization term~\cite{FWI-with-TV-constraint,FWI-with-TV-constraint2,FWI-with-TV-constraint3,FWI-with-TV-constraint4}.
This formulation decouples the parameter for the TV constraint from the FWI objective, allowing it to be determined independently based on prior knowledge of subsurface properties~\cite{constraint0,constraint1,constraint2,constraint3,constraint4,constraints-vs-penalties-in-FWI}.
By doing so, this approach not only simplifies parameter selection but also enhances the interpretability of both the mathematical formulation and the reconstructed subsurface models.

Despite its advantages, solving the TV-constrained FWI problem presents considerable difficulties due to the interplay between the nonlinear observation process and the non-smoothness of the TV term.
Conventional methods~\mbox{\cite{FWI-with-TV-constraint,FWI-with-TV-constraint2,FWI-with-TV-constraint3,FWI-with-TV-constraint4}} attempt to address these issues by incorporating inner loops to enforce the constraint at each optimization step or by employing linear or quadratic approximations.
However, these approaches come with notable drawbacks: inner loops substantially increase computational cost, and approximations compromise reconstruction accuracy.
This raises a crucial question: \textit{Is it possible to develop an algorithm that solves the TV-constrained FWI problem efficiently while avoiding inner loops and approximations?}

In this paper, we introduce a novel algorithm for solving the TV-constrained FWI problem using a primal-dual splitting method~\cite{PDS2}.
The proposed algorithm effectively addresses the intertwined issues of the nonlinear observation process and the non-smoothness of the TV constraint, achieving accurate reconstructions without relying on approximations.
Additionally, by eliminating the need for inner loops, our approach is markedly more computationally efficient than existing methods.
We validate the performance of our algorithm through numerical experiments on the SEG/EAGE Salt and Overthrust Models, demonstrating its capability to efficiently enforce the constraints while delivering high-quality reconstructions.
